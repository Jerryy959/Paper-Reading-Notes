\documentclass[10pt]{article}
\usepackage{NotesTeXV3} %/Path/to/package should be replaced with package location
\usepackage{lipsum}
\usepackage{ctex}
\renewcommand{\contentsname}{Contents}
\title{{\Huge Paper Reading Notes}\\{\Large{Jerry}}}
\author{Junyi Yang\footnote{\href{https://github.com/Jerryy959}{\textit{Jerryy959}}}}

\affiliation{PhD. Student at the Institute of Computing Technology, Chinese Academy of Sciences\\
\href{https://github.com/Jerryy959}{Website}\\
\href{https://github.com/Jerryy959}{LinkedIn}\\
\href{https://github.com/Jerryy959}{Github}\\
}
\emailAdd{yangjunyi22s@ict.ac.cn}


\begin{document}
  \maketitle
  \flushbottom
  \newpage
  \pagestyle{fancynotes}
  \part{介绍}

  \section{规划}\label{sec:Introduction}
  这份笔记主要记录自己阅读的paper,与对应的总结。分类原则是按照类别进行划分,主要包括体系结构与大模型。

  \newpage

  \part{文章}

  \section{计算机体系结构}\label{sec:Architecture}

  \newpage

  \section{人工智能与大模型}\label{sec:Ai}
%   \lipsum[1]
%   \section{Subtest Section 1}
%   Some notes here.\sn{With some additional sidenotes}
\end{document}