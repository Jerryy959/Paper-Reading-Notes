\documentclass[10pt]{article}
\usepackage{NotesTeXV3} %/Path/to/package should be replaced with package location
\usepackage{lipsum}
\usepackage{ctex}
\renewcommand{\CJKfamilydefault}{\CJKsfdefault}
\renewcommand{\contentsname}{Contents}
\title{{\Huge Paper Reading Notes}\\{\Large{Jerry}}}
\author{Junyi Yang\footnote{\href{https://github.com/Jerryy959}{\textit{Jerryy959}}}}

\affiliation{PhD. Student at the Institute of Computing Technology, Chinese Academy of Sciences\\
\href{https://github.com/Jerryy959}{Website}\\
\href{https://github.com/Jerryy959}{LinkedIn}\\
\href{https://github.com/Jerryy959}{Github}\\
}
\emailAdd{yangjunyi22s@ict.ac.cn}


\begin{document}
  \maketitle
  \flushbottom
  \newpage
  \pagestyle{fancynotes}
  \part{介绍}

  \section{大纲}\label{sec:Introduction}
  这份笔记主要记录自己阅读的paper,与对应的总结。分类原则是按照类别进行划分,主要包括体系结构与大模型。

  \newpage

  \part{文章}

  \section{计算机体系结构}\label{sec:Architecture}
  \subsection{SpecLFB: Eliminating Cache Side Channels in Speculative Executions\cite{294605}}\label{usenixsecurity24-cheng-xiaoyu}

  \begin{quotation}
    \textit{Cache side-channel attacks based on speculative executions
    are powerful and difficult to mitigate. Existing hardware defense
    schemes often require additional hardware data structures,
    data movement operations and/or complex logical computations,
    resulting in excessive overhead of both processor
    performance and hardware resources. To this end, this paper
    proposes SpecLFB, which utilizes the microarchitecture component,
    Line-Fill-Buffer, integrated with a proposed mechanism
    for load security check to prevent the establishment of
    cache side channels in speculative executions. To ensure the
    correctness and immediacy of load security check, a structure
    called ROB unsafe mask is designed for SpecLFB to track
    instruction state. To further reduce processor performance
    overhead, SpecLFB narrows down the protection scope of
    unsafe speculative loads and determines the time at which
    they can be deprotected as early as possible. SpecLFB has
    been implemented in the open-source RISC-V core, Sonic-
    BOOM, as well as in Gem5. For the enhanced SonicBOOM,
    its register-transfer-level (RTL) code is generated, and an
    FPGA hardware prototype burned with the core and running a
    Linux-kernel-based operating system is developed. Based on
    the evaluations in terms of security guarantee, performance
    overhead, and hardware resource overhead through RTL simulation,
    FPGA prototype experiment, and Gem5 simulation,
    it shows that SpecLFB effectively defends against attacks. It
    leads to a hardware resource overhead of only 0.6\% and the
    performance overhead of only 1.85\% and 3.20\% in the FPGA
    prototype experiment and Gem5 simulation, respectively.}
  \end{quotation}

  \begin{remark}\textbf{翻译:}
    \textit{缓存侧信道攻击\mn{利用Cache信息获取信息,见\href{https://blog.csdn.net/qq_41691212/article/details/133847749}{常见的侧信道攻击方法}}基于推测执行非常强大且难以缓解。现有的硬件防御方案通常需要额外的硬件数据结构、数据移动操作和/或复杂的逻辑计算,导致处理器性能和硬件资源的过度开销。为此,本文提出了SpecLFB,它利用微架构组件,即行填充缓冲区(Line-Fill-Buffer),并集成了用于加载安全检查的提出机制,以防止在推测执行中建立缓存侧信道。为确保加载安全检查的正确性和即时性,为SpecLFB设计了一种称为ROB不安全掩码的结构,用于跟踪指令状态。为了进一步减少处理器性能开销,SpecLFB缩小了不安全推测加载的保护范围,并尽可能早地确定它们可以解除保护的时间。SpecLFB已在开源RISC-V核心Sonic-BOOM以及Gem5中实现。对于增强的SonicBOOM,生成了其寄存器传输级(RTL)代码,并开发了一个烧录了核心并运行基于Linux内核操作系统的FPGA硬件原型。基于通过RTL模拟、FPGA原型实验和Gem5模拟在安全保证、性能开销和硬件资源开销方面的评估,它表明SpecLFB有效地防御了攻击。在FPGA原型实验和Gem5模拟中,它导致硬件资源开销仅为0.6\%,性能开销分别为1.85\%和3.20\%。} 
    \end{remark}
    

  \newpage

  \section{人工智能与大模型}\label{sec:Ai}
%   \lipsum[1]
%   \section{Subtest Section 1}
%   Some notes here.\sn{With some additional sidenotes}

    \newpage

    \bibliographystyle{plain}
    \bibliography{bibliography}
    \addcontentsline{toc}{section}{References}

\end{document}